%%%%%%%%%%%%%%%%%%% vorlage.tex %%%%%%%%%%%%%%%%%%%%%%%%%%%%%
%
% LaTeX-Vorlage zur Erstellung von Projekt-Dokumentationen
% im Fachbereich Informatik der Hochschule Trier
%
% Basis: Vorlage svmono des Springer Verlags
%
%%%%%%%%%%%%%%%%%%%%%%%%%%%%%%%%%%%%%%%%%%%%%%%%%%%%%%%%%%%%%

\documentclass[envcountsame,envcountchap, deutsch]{i-studis}

\usepackage{makeidx}         	% Index
\usepackage{multicol}        	% Zweispaltiger Index
%\usepackage[bottom]{footmisc}	% Erzeugung von Fußnoten

%%-----------------------------------------------------
%\newif\ifpdf
%\ifx\pdfoutput\undefined
%\pdffalse
%\else
%\pdfoutput=1
%\pdftrue
%\fi
%%--------------------------------------------------------
%\ifpdf
\usepackage[pdftex]{graphicx}
\usepackage{epstopdf}
\usepackage[pdftex,plainpages=false]{hyperref}
\usepackage{subcaption}
%\else
%\usepackage{graphicx}
%\usepackage[plainpages=false]{hyperref}
%\fi

%%-----------------------------------------------------
\usepackage{color}				% Farbverwaltung
%\usepackage{ngerman} 			% Neue deutsche Rechtsschreibung
\usepackage[english, ngerman]{babel}
%\usepackage[latin1]{inputenc} 	% Ermöglicht Umlaute-Darstellung
\usepackage[utf8]{inputenc}  	% Ermöglicht Umlaute-Darstellung unter Linux (je nach verwendetem Format)
\usepackage[T1]{fontenc}
\usepackage{textcomp}
\usepackage{gensymb}
%-----------------------------------------------------
\usepackage{listings} 			% Code-Darstellung
\lstset
{
	basicstyle=\scriptsize, 	% print whole listing small
	keywordstyle=\color{blue}\bfseries,
								% underlined bold black keywords
	identifierstyle=, 			% nothing happens
	commentstyle=\color{red}, 	% white comments
	stringstyle=\ttfamily, 		% typewriter type for strings
	showstringspaces=false, 	% no special string spaces
	framexleftmargin=7mm, 
	tabsize=3,
	showtabs=false,
	frame=single, 
	rulesepcolor=\color{blue},
	numbers=left,
	linewidth=146mm,
	xleftmargin=8mm
}
\usepackage{textcomp} 			% Celsius-Darstellung
\usepackage{amssymb,amsfonts,amstext,amsmath}	% Mathematische Symbole
\usepackage[german, ruled, vlined]{algorithm2e}
\usepackage[a4paper]{geometry} % Andere Formatierung
\usepackage{bibgerm}
\usepackage{array}
\hyphenation{Ele-men-tar-ob-jek-te  ab-ge-tas-tet Aus-wer-tung House-holder-Matrix Le-ast-Squa-res-Al-go-ri-th-men} 		% Weitere Silbentrennung bei Bedarf angeben
\setlength{\textheight}{1.1\textheight}
\pagestyle{myheadings} 			% Erzeugt selbstdefinierte Kopfzeile
\makeindex 						% Index-Erstellung


%--------------------------------------------------------------------------
\begin{document}
%------------------------- Titelblatt -------------------------------------
\title{Prozedurale Generierung von Baumstrukturen innerhalb der Unreal Engine 4}
\subtitle{Procedural generation of tree-like structures in the Unreal Engine 4}
%---- Die Art der Dokumentation kann hier ausgewählt werden---------------
%\project{Bachelor-Projektarbeit}
\project{Bachelor-Abschlussarbeit}
%\project{Master-Projektstudium}
%\project{Master-Abschlussarbeit}
%\project{Seminar zur Vorlesung ...}
%\project{Hausarbeit zur Vorlesung ...}
%--------------------------------------------------------------------------
\supervisor{\\Prof. Dr. Christof Rezk-Salama} 		% Betreuer der Arbeit
\author{David Liebemann} 							% Autor der Arbeit
\address{Trier,} 							% Im Zusammenhang mit dem Datum wird hinter dem Ort ein Komma angegeben
\submitdate{26.02.2017} 				% Abgabedatum
%\begingroup
%  \renewcommand{\thepage}{title}
%  \mytitlepage
%  \newpage
%\endgroup
\begingroup
  \renewcommand{\thepage}{Titel}
  \mytitlepage
  \newpage
\endgroup
%--------------------------------------------------------------------------
\frontmatter 
%--------------------------------------------------------------------------
%\input{chapters/Vorwort}				% Vorwort (optional)
\kurzfassung

In der folgenden Abschlussarbeit werden zwei verschiedene Verfahren zur prozeduralen Generierung von Baumstrukturen und die Implementierungen dieser innerhalb des Frameworks der Unreal Engine 4 vorgestellt. Basierend auf der Untersuchung bisheriger Arbeiten werden zwei Ansätze gewählt: Lindenmayer-Systeme und ein Space Colonization Algorithmus.

Ein Lindenmayer-System -- kurz: L-System -- ist eine Erweiterung von kontextfreien Grammatiken, welche Teile einer übergebenen Zeichenkette anhand festgelegter Regeln durch andere Zeichenketten ersetzen. \cite[S.2]{ABOP:04} Die grundlegende Funktionsweise und die für eine Generierung von Baumstrukturen benötigten Erweiterungen werden behandelt sowie die verwendete Methode zur Visualisierung der Ergebnisse von L-Systemen vorgestellt.

Das Prinzip des verwendeten Space Colonization Algorithmus basiert auf einer biologisch motivierten Simulation der Konkurrenz von wachsenden Zweigen um Wachstumsraum. \cite[S.2f]{SpaceColonizationAlgorithm:07} Benötigte Eingaben, der Ablauf des Algorithmus und die Prozedur zur Generierung von Baumstrukturen werden erläutert sowie Erweiterungen des ursprünglichen Algorithmus vorgestellt.

Die im Rahmen dieser Arbeit umgesetzten Implementierungen beider Verfahren werden behandelt. Die verwendete Datenstruktur in Form eines graphentheoretischen Baumes und das darauf basierende Modellgenerierungssystem, welches für die Konstruktion der Modelldaten verantwortlich ist, werden vorgestellt. Die aus den Implementierungen resultierenden Baumstrukturen werden präsentiert und der Einfluss von Parametern auf das visuelle Erscheinungsbild sowie die Effizienz der Generierung der Modelle wird behandelt. 

Abschließend findet eine Bewertung und ein Vergleich beider Verfahren auf Grundlage der vorgestellten Ergebnisse sowie die Behandlung wünschenswerter Erweiterungen für zukünftige Arbeiten statt.
 \\
 \\
 
 In the following bachelor thesis we present two different methods for the procedural generation of tree-like structures and their implementations in the framework of the Unreal Engine 4. Based on the analysis of past works we chose two approaches: Lindenmayer-Systems and a Space Colonization Algorithm.
 
 A Lindenmayer-System -- in short: L-System -- is an extension of context free grammars which replace parts of a given string by other strings, according to a predefined ruleset. \cite[S.2]{ABOP:04} We examine the basic functionality and the extensions required for generating  tree-like structures and present the chosen visualization method for displaying the results.
 
 The functionality of the Space Colonization Algorithm is biologically motivated by the competition for space between growing branches. We introduce the necessary input for the algorithm, the algorithmic process, the procedure for generating tree-like structures and extensions of the original algorithm.
 
 We present the implementations of both techniques, the graph-theoretical tree data structure used by us and the system responsible for constructing the 3D model data. Afterwards we display the resulting tree structures and discuss the impact of parameter values on the visual appearance and efficiency of the generation methods.
 
 Finally we assess and compare both techniques based on the presented results and discuss desirable extensions reserved for future works.


 			% Kurzfassung Deutsch/English
\tableofcontents 						% Inhaltsverzeichnis
\listoffigures 							% Abbildungsverzeichnis (optional)
%\listoftables 							% Tabellenverzeichnis (optional)
%--------------------------------------------------------------------------
\mainmatter                        		% Hauptteil (ab hier arab. Seitenzahlen)
%--------------------------------------------------------------------------
% Die Kapitel werden in separaten .tex-Dateien abgelegt und hier eingebunden.
\chapter{Einleitung und Problemstellung}
\cite{ABOP:04}

\chapter{Lindenmayer-Systeme}

Vorstellung von Konzepten für L-Systeme

\section{Definition}

\subsection{D0L-Systeme}
\subsection{Parametrische L-Systeme}
\subsection{Verzweigte L-Systeme}

\section{Turtle-Grafik}

\section{L-Systeme für Baumstrukturen}

\section{Einfluss der Parameter}
\chapter{Space Colonization Algorithmus} \label{ch:SCA}

Es werden die benötigten Eingaben und der Ablauf des Space Colonization Algorithmus behandelt. Die Prozedur zur Generierung von Baumstrukturen wird vorgestellt und die implementierten Erweiterungen des ursprünglichen Algorithmus werden erläutert.

\section{Ursprung}
Der vorgestellte Space Colonization Algorithmus wurde ursprünglich zur Modellierung und Visualisierung von Blattvenen entwickelt und basiert auf der Wirkung des Pflanzenhormons Auxin. Dieser Hormonstoff entsteht im Blatt und wird von bereits existierenden Blattvenen angezogen, der resultierende Hormonstrom führt zur Bildung von neuen Venen im Blatt. Die Simulation dieses Vorgangs führt zu realitätsnahen Venenmustern. \cite[Abschn. 2.5]{LeafVenation:05}

Mithilfe einer Erweiterung in den dreidimensionalen Raum und Nachbearbeitung der Resultate kann eine Vielfalt von Baum- und Strauchstrukturen generiert werden. \cite[Abschn. 1]{SpaceColonizationAlgorithm:07}

\section{Aufbau}

Der Algorithmus verarbeitet eine Menge von Einflusspunkten $S$ und baut darauf basierend einen Baum $G = \langle V,E \rangle$ auf. 

Der Algorithmus benötigt die folgenden Eingaben:

\begin{description}[labelindent]
	\item[\boldmath$d_i$] Der Einflussradius. Einflusspunkte prägen den Aufbau des Baums nur, wenn sich Knotenpunkte innerhalb dieses Radius befinden. \cite[Abschn. 2]{SpaceColonizationAlgorithm:07}\\
	
	\item[\boldmath$d_k$] Der Minimalradius. Befindet sich ein Knotenpunkt innerhalb des Minimalradius um einen Einflusspunkt, wird dieser aus der Menge der Einflusspunkte $S$ entfernt. \cite[Abschn. 2]{SpaceColonizationAlgorithm:07}\\
	
	\item[\boldmath$D$] Die Schrittweite. Jeder neu generierte Knotenpunkt wird in diesem Abstand zu seinem Vorgänger positioniert. \cite[Abschn. 2]{SpaceColonizationAlgorithm:07} \\
	
	\item[\boldmath$\overrightarrow{T}$] Der Tropismusvektor. 

\end{description}

\section{Ablauf}
\label{sec:SCA_Ablauf}
Zu Beginn des Algorithmus werden $N$ Einflusspunkte in einem vorgegebenen Bereich generiert. Dieser Einflussbereich signalisiert die Verfügbarkeit von Raum, in dem die Baumstruktur entstehen kann.  \cite[Abschn. 2]{SpaceColonizationAlgorithm:07}

Daraufhin wird der Baum iterativ aufgebaut und durchläuft in jeder Iteration die folgenden Schritte: 

\begin{description}[labelindent]
	\item[\boldmath$1.$] Für jeden Einflusspunkt in $S$ wird der am nächsten liegende Knotenpunkt $v\in V$ bestimmt. Befindet sich $v$ innerhalb des Einflussradius $d_i$ eines Einflusspunktes, wird der Knotenpunkt einer zugeordneten Menge $S(v)$ hinzugefügt. $S(v)$ beinhaltet somit alle Einflusspunkte, die einen Einfluss auf den Knotenpunkt ausüben. \cite[Abschn. 2]{SpaceColonizationAlgorithm:07} \label{alg:SCA_1}\\
	
	\item[\boldmath$2.$] Befinden sich Elemente in $S(v)$, wird ein neuer Knotenpunkt $v_n$ den Nachfolgern von $v$ hinzugefügt und $v$ als Vorgänger von $v_n$ eingetragen.  Alle Punkte in $S(v)$ beeinflussen $v_n$ in gleichem Maße, die neue Position $\overrightarrow{p_{v_n}}$ des Knotenpunkts berechnet sich somit wie folgt:
	
	\begin{equation}
	\begin{array}{ll}
	\overrightarrow{p_{v_n}} & = \overrightarrow{p_v} + D * \overrightarrow{n_{T}}
	\end{array}
	\end{equation} 
	
	\begin{equation}
	\begin{array}{ll}
	\text{  mit  } \overrightarrow{n_{T}} & = \dfrac{\overrightarrow{n} + \overrightarrow{T} }{\lVert\overrightarrow{n} + \overrightarrow{T}\rVert}  
	\end{array}
	\end{equation} 
	
	\begin{equation}
	\begin{array}{ll}
	\text{ und }  \overrightarrow{n} = \sum\limits_{s \in S(v)}\dfrac{\overrightarrow{p_s} - \overrightarrow{p_v}}{\lVert \overrightarrow{p_s} - \overrightarrow{p_v} \rVert}
	\end{array}
	\end{equation}	
	\cite[Abschn. 2]{SpaceColonizationAlgorithm:07} \label{alg:SCA_2}\\
	
	\item[\boldmath$3.$] Für jeden Einflusspunkt wird überprüft, ob sich ein Knotenpunkt innerhalb des Minimalradius $d_k$ befindet. Existiert ein solcher Knotenpunkt, wird der Einflusspunkt aus der Menge der Einflusspunkte $S$ entfernt. \cite[Abschn. 2]{SpaceColonizationAlgorithm:07} \label{alg:SCA_3}
\end{description}

Diese Schritte werden solange ausgeführt, bis alle Einflusspunkte entfernt wurden, sich kein Knotenpunkt im Einflussradius eines Einflusspunktes befindet oder bis eine vorgegebene Maximalanzahl von Iterationen durchgeführt wurde. Abbildung \ref{fig:SCA_Basic} zeigt eine beispielhafte Anwendung des Algorithmus.

\begin{figure} [hbtp]
	\centering
	\begin{subfigure}[t]{.31\textwidth}
		\centering
		\includegraphics[width=\linewidth]{images/SCA_Basic1.png}
		\caption{ Ausgangssituation. }
		\label{subfig:SCA_Basic1}
	\end{subfigure}
	\hspace{.01\textwidth}
	\begin{subfigure}[t]{.31\textwidth}
		\centering
		\includegraphics[width=\linewidth]{images/SCA_Basic2.png}
		\caption{Schritt 1: Blaue Kreise entsprechen dem Einflussradius.}
		\label{subfig:SCA_Basic2}
	\end{subfigure}
	\hspace{.01\textwidth}
	\begin{subfigure}[t]{.31\textwidth}
		\centering
		\includegraphics[width=\linewidth]{images/SCA_Basic3.png}
		\caption{Schritt 2: Bestimmung von $\tilde{n}$ (roter Pfeil) ohne Einbeziehung von $\overrightarrow{T}$.}
		\label{subfig:SCA_Basic3}
	\end{subfigure}

	\begin{subfigure}[t]{.31\textwidth}
		\centering
		\includegraphics[width=\linewidth]{images/SCA_Basic4.png}
		\caption{Schritt 2: Platzierung des neuen Knotenpunkts.}
		\label{subfig:SCA_Basic4}
	\end{subfigure}
	\hspace{.01\textwidth}
	\begin{subfigure}[t]{.31\textwidth}
		\centering
		\includegraphics[width=\linewidth]{images/SCA_Basic5.png}
		\caption{Schritt 3: Rote Kreise entsprechen dem Minimalradius.}
		\label{subfig:SCA_Basic5}
	\end{subfigure}
	\hspace{.01\textwidth}
	\begin{subfigure}[t]{.31\textwidth}
		\centering
		\includegraphics[width=\linewidth]{images/SCA_Basic6.png}
		\caption{Ausgangssituation der nächsten Iteration.}
		\label{subfig:SCA_Basic6}
	\end{subfigure}

	\begin{subfigure}[t]{.31\textwidth}
		\centering
		\includegraphics[width=\linewidth]{images/SCA_Basic7.png}
		\caption{Einflusspunkte beeinflussen unterschiedliche Knotenpunkte.}
		\label{subfig:SCA_Basic7}
	\end{subfigure}
	\hspace{.01\textwidth}
	\begin{subfigure}[t]{.31\textwidth}
		\centering
		\includegraphics[width=\linewidth]{images/SCA_Basic8.png}
		\caption{Eine Verzweigung entsteht.}
		\label{subfig:SCA_Basic8}
	\end{subfigure}
	\caption{Beispielhafte Anwendung des Space Colonization Algorithmus. Blaue Punkte entsprechen Einflusspunkten, weiße Punkte entsprechen Knotenpunkten. Der unterste Knotenpunkt stellt die Wurzel dar. Eigene Abbildungen auf Grundlage von \cite[Abb. 2]{SpaceColonizationAlgorithm:07}.}\label{fig:SCA_Basic}
	
\end{figure}


\section{Generierung von Baumstrukturen}
\label{sec:GenerierungBaumstrukturen}
Der Space Colonization Algorithmus liefert einen Baum, der Knotenpunkte enthält, welche Positionen im dreidimensionalen Raum darstellen. Um diese Knotenpunkte in Form von baumähnlichen Strukturen zu visualisieren, wird die Prozedur erweitert. Der Aufbau der Baumstruktur läuft wie folgt ab:

\begin{description}[labelindent]
	\item[\boldmath$1.$] Der Einflussbereich wird mit der vorgegeben Anzahl von Einflusspunkten gefüllt. \cite[Abschn. 2]{SpaceColonizationAlgorithm:07} \\

	\item[\boldmath$2.$] Der Baum wird, wie in Abschnitt \ref{sec:SCA_Ablauf} beschrieben, iterativ generiert. \cite[Abschn. 2]{SpaceColonizationAlgorithm:07} \\

	\item[\boldmath$3.$] Die Nachfolger jedes Knotenpunkts werden einander angenähert, um eine Verringerung der Abzweigungswinkel zwischen den verbindenden Kanten zu erreichen. Dies führt zu einer insgesamt realistischeren Baumstruktur. \cite[Abschn. 2]{SpaceColonizationAlgorithm:07} \\
	
	\item[\boldmath$4.$] Die Kanten, welche die Knotenpunkte verbinden, werden mithilfe von Zylindern visualisiert, um die Aststruktur eines biologischen Baumes zu simulieren. \cite[Abschn. 2]{SpaceColonizationAlgorithm:07} 
	
\end{description}

Die von Runions u.a. \cite{SpaceColonizationAlgorithm:07} vorgeschlagene Kurven-Unterteilung \cite[Abschn. 2]{SpaceColonizationAlgorithm:07} wurde in dieser Arbeit nicht behandelt, da durch Angabe der Schrittweite $D$ eine ausreichende visuelle Qualität erzielt wurde.

Abbildung \ref{fig:SCA_Extended} zeigt die Modellierung einer zweidimensionalen Baumstruktur.

\paragraph{Verringerung der Abzweigungswinkel} \label{par:VerringerungAbzweigungswinkel}

Die $m$ Nachfolger $v_{1} ... v_{m}$ eines Knotens $v$ werden wie folgt einander angenähert:
\begin{equation}
\overrightarrow{m} = \overrightarrow{p_v} + \dfrac{\sum_{i=0}^{m} (\overrightarrow{p_{v_i}} - \overrightarrow{p_v})}{m}
\end{equation}

wobei $\overrightarrow{m}$ dem arithmetischen Mittelpunkt der Nachfolger entspricht. Die Positionen der Nachfolger werden nun wie folgt dem Mittelpunkt angenähert:

\begin{equation}
\overrightarrow{p_{v_i}} = \overrightarrow{p_{v_i}} + \dfrac{\overrightarrow{m} -\overrightarrow{p_{v_i}}}{2}
\end{equation}


\paragraph{Berechnung der Zylinderbreiten}

Die Berechnung der Zylinderbreiten erfolgt anhand von Murrays Regel, die besagt, dass der Radius $r$ eines Astes auf Grundlage der Radien $r_{n_1}...r_{n_m}$ der nachfolgenden, von ihm abzweigenden Äste wie folgt berechnet werden kann: 

\begin{equation}
\begin{array}{ll}
r^g & = r_{n_1}^g + r_{n_2}^g + ... + r_{n_m}^g 
\end{array}
\label{eq:Radiusberechnung}
\end{equation} 

Diese Berechnung kann rekursiv auf dem Baum ausgeführt werden, indem für jeden Knotenpunkt der Radius aus den Radien seiner Nachfolger berechnet wird. Besitzt ein Knotenpunkt keine Nachfolger, wird ein festgelegter Radius $r_0$ zurückgegeben. \cite[Abschn. 3.5]{LeafVenation:05} Der Wert $g$ kann frei gewählt werden.

\begin{figure} [hbtp]
	\centering
	\begin{subfigure}[t]{.4\textwidth}
		\centering
		\includegraphics[width=\linewidth]{images/SCA_Extended1.png}
		\caption{$N=2000$ Einflusspunkte, zufällig in einem Ring mit dem Radius $r = 500$ verteilt.}
		\label{subfig:SCA_Extended1}
	\end{subfigure}
	\hspace{.1\textwidth}
	\begin{subfigure}[t]{.4\textwidth}
		\centering
		\includegraphics[width=\linewidth]{images/SCA_Extended2.png}
		\caption{Ergebnis des Space Colonization Algorithmus mit einem Einflussradius $d_i = 100$, Minimalradius $d_k = 20$, Schritteweite $D = 15$ und $\overrightarrow{T} = \overrightarrow{0}$.}
		\label{subfig:SCA_Extended2}
	\end{subfigure}
	\begin{subfigure}[t]{.4\textwidth}
		\centering
		\includegraphics[width=\linewidth]{images/SCA_Extended3.png}
		\caption{Verringerung der Abzweigungswinkel.}
		\label{subfig:SCA_Extended3}
	\end{subfigure}
	\hspace{.1\textwidth}
	\begin{subfigure}[t]{.4\textwidth}
		\centering
		\includegraphics[width=\linewidth]{images/SCA_Extended4.png}
		\caption{Modellierung des Ergebnis mithilfe von Zylindern mit $r_0 = 1$ und $g=2$.}
		\label{subfig:SCA_Extended4}
	\end{subfigure}
	\caption{Modellierung einer zweidimensionalen Baumstruktur entsprechend der in Abschnitt \ref{sec:GenerierungBaumstrukturen} beschriebenen Schritte. Eigene Abbildungen.}
	\label{fig:SCA_Extended}
\end{figure}

\section{Erweiterungen} \label{sec:SCA_Erweiterungen}

Im Rahmen dieser Arbeit wurden die folgenden Erweiterungen des ursprünglichen Algorithmus entwickelt und implementiert.

\paragraph{Zweigtiefe}

Die Zweigtiefe $Z(v)$ eines Knotens $v \in V$ wird für Erweiterungen des Space Colonization Algorithmus und die Generierung von Modelldaten verwendet. Sie entspricht nicht der Tiefe des Knotens und berechnet sich wie folgt:
\begin{equation}
Z(v)= \begin{cases}
0 & \text{falls v die Wurzel ist} \\
Z(v') & \text{falls v' genau einen Nachfolger besitzt}\\
1 + Z(v') & \text{sonst}
\end{cases} 
\end{equation}
wobei $v'$ den Vorgänger des Knotens $v$ darstellt. Die Zweigtiefe wird direkt nach der Erstellung des Knotens berechnet und daraufhin nicht verändert. Ein Beispiel für die Bestimmung der Zweigtiefe wird in Abbildung \ref{fig:SCA_Zweigtiefe} gezeigt.

Der von Prusinkiewicz u.a. \cite{ABOP:04} beschriebene Begriff von axialen Bäumen \cite[S.21]{ABOP:04} und die zugehörige Bestimmung der Achsentiefe unterscheidet sich von der Berechnung der Zweigtiefe. Knoten können dieselbe Zweigtiefe besitzen, selbst wenn ihre Positionen keine gerade Linie bilden.

\begin{figure} [hbtp]
	\centering
	\begin{subfigure}[t]{.31\textwidth}
		\centering
		\includegraphics[width=\linewidth]{images/SCA_Zweigtiefe1.png}
		\caption{Ausgangssituation.}
		\label{subfig:SCA_Zweigtiefe1}
	\end{subfigure}
	\hspace{.01\textwidth}
	\begin{subfigure}[t]{.31\textwidth}
		\centering
		\includegraphics[width=\linewidth]{images/SCA_Zweigtiefe2.png}
		\caption{Der Vorgänger des neuen Knotens besitzt genau einen Nachfolger.}
		\label{subfig:SCA_Zweigtiefe2}
	\end{subfigure}
	\hspace{.01\textwidth}
	\begin{subfigure}[t]{.31\textwidth}
		\centering
		\includegraphics[width=\linewidth]{images/SCA_Zweigtiefe3.png}
		\caption{Die Zweigtiefe des rechten neuen Knotens ist um $1$ größer als der seines Vorgängers, da dieser mehr als einen Nachfolger besitzt.}
		\label{subfig:SCA_Zweigtiefe3}
	\end{subfigure}
	\caption{Darstellung der Zweigtiefen von Knotenpunkten aus Abbildung \ref{fig:SCA_Basic}. }
	\label{fig:SCA_Zweigtiefe}
\end{figure}

\paragraph{Zusätzliche Bedingungen}

Um die vom Space Colonization Algorithmus generierte Baumstruktur besser kontrollieren zu können, wurden zusätzliche Bedingungen zum Ablauf des Algorithmus hinzugefügt:
\begin{description}[labelindent]
	\item[\boldmath$max_{grad}$] Der maximale Grad der Knotenpunkte im Baum. Bevor ein neuer Knotenpunkt $v_n$ als Nachfolger eines Knotenpunkts $v$ erstellt wird, wird die Anzahl der Nachfolger von $v$ überprüft. Entspricht diese $max_{grad}$, wird der Knotenpunkt $v_n$ nicht erstellt.\\
	
	\item[\boldmath$max_{Z}$] Die maximale Zweigtiefe eines Knotens. Bevor ein neuer Knotenpunkt $v_n$ als Nachfolger eines Knotenpunkts $v$ erstellt wird, wird seine Zweigtiefe $Z(v_n)$ berechnet. Übersteigt diese $max_{Z}$, wird der Knotenpunkt $v_n$ nicht erstellt.\\
	
	\item[\boldmath$max_{NG}$] Die maximale Anzahl von Iterationen, in welchen einem Knotenpunkt kein neuer Nachfolger hinzugefügt wurde.\footnote{$NG$ steht für \glqq No Grow\grqq{} oder \glqq Did not grow\grqq, englisch für \glqq kein Wachstum\grqq.} Hat ein Knoten diesen Wert erreicht, wird er in Schritt 1 des Algorithmus \ref{alg:SCA_1} nicht weiter darauf untersucht, ob er sich innerhalb des Einflussradius eines Einflusspunktes befindet. Der Wert von $max_{NG}$ eines Knotenpunkts wird auf $0$ zurückgesetzt, falls diesem ein neuer Nachfolger hinzugefügt wurde.
	
	Durch die vorsichtige Wahl von $max_{NG}$ können Positionsvergleiche vermieden werden ohne den resultierenden Baum zu verändern.
\end{description}

\paragraph{Gewichtetes Wachstum}

Gewichtetes Wachstum ist eine Option die Schrittweite $D$ in Abhängigkeit von der Zweigtiefe zu erhöhen. Befindet sich ein Knotenpunkt auf der Zweigtiefe $0$, werden neu hinzugefügte Nachfolger im Abstand von $2 * D$ positioniert, befindet er sich auf der maximalen Zweigtiefe, werden neu hinzugefügte Nachfolger im Abstand von $D$ positioniert. Zwischen diesen beiden Zweigtiefen wird die Schrittweite linear interpoliert.

\paragraph{Kurvenreduktion} \label{par:Kurvenreduktion}
 
Kurvenreduktion in Abhängigkeit des Abzweigungswinkels ermöglicht es die Baumstruktur mithilfe einer verringerten Datenmenge darzustellen. Besitzt ein Knoten $v$ genau einen Nachfolger $v_n$ und einen Vorgänger $v'$, werden die zwei Richtungsvektoren zwischen $v'$ und $v$ sowie $v$ und $v_n$ berechnet:
\begin{equation}
\overrightarrow{R_1} = \dfrac{\overrightarrow{p_v} - \overrightarrow{p_{v'}}}{\lVert\overrightarrow{p_v} - \overrightarrow{p_{v'}} \rVert} \text{ und } \overrightarrow{R_2} = \dfrac{\overrightarrow{p_{v_n}} - \overrightarrow{p_{v}}}{\lVert \overrightarrow{p_{v_n}} - \overrightarrow{p_{v}} \rVert}
\end{equation}

Übersteigt das Skalarprodukt $\langle \overrightarrow{R_1}, \overrightarrow{R_2} \rangle$ einen festgelegten Maximalwert $max_K$ zwischen $0$ und $1$, wird $v$ aus dem Baum entfernt. $v_n$ wird zum Nachfolger von $v'$ und $v'$ zum Vorgänger von $v_n$. 

Die Kurvenreduktion kann auch auf Bäume angewendet werden, welche wie in Abschnitt \ref{sec:LS_Baumstrukturen} beschrieben durch eine Turtle-Interpretation aufgebaut wurden.
\chapter{Implementierung} \label{ch:Implementierung}


Im folgenden Kapitel wird die Implementierung der Vorgehen innerhalb des Frameworks der Unreal Engine 4 behandelt. Die Baumrepräsentation enthält Daten, die von den L-System und Space Colonization Implementierungen generiert werden. Diese Daten werden an das Modellgenerierungssystem übergeben, welches die Modelldaten für eine grafische Darstellung in der Unreal Engine 4 produziert.


\section{Baumrepräsentation}  \label{sec:ImplementierungBaumrep}

Sowohl L-Systeme als auch der Space Colonization Algorithmus generieren einen graphentheoretischen Baum, auf dessen Grundlage die Modellgenerierung durchgeführt wird. Die implementierte Baumrepräsentation kann daher von beiden Systemen verwendet werden und ermöglicht es, diese mit demselben Modellgenerierungssystem zu visualisieren.

Der Baum wird durch eine Datenklasse repräsentiert, jedes Objekt dieser Klasse beschreibt einen Knoten sowie die Kante, welche vom Vorgänger zu dem Knoten führt. Die Datenklasse bietet Zugriff auf die folgenden Informationen:

\begin{description}
	\item \textbf{Vorgänger und Nachfolger:} Mithilfe eines Verweises auf den Vorgänger und eine Liste der Nachfolger eines Knotens kann der Baum-Graph vollständig repräsentiert werden. Weiterhin ermöglicht dies die Implementierung einer Reihe von rekursiven Funktionen zur Anpassung von Modelldaten.\\
	
	\item \textbf{Modell-Daten:} Kanten werden, wie in Abschnitt \ref{subsec:ZylinderMeshes} beschrieben, mithilfe von Zylindern visualisiert. Um die Generierung von Modelldaten zu vereinfachen, bietet die Datenklasse Zugriff auf Start- und Endposition, Start- und Endradius, Start- und Endnormale sowie einen Rotationswinkel. 
	
	Des Weiteren wird die Zweigtiefe des repräsentierten Knoten gespeichert.\\
	
	\item \textbf{Wachstums-Daten:} Die Wachstums-Daten bestehen aus einer Wachstumsrichtung, einem Einfluss-Zähler und dem \glqq Kein Wachstum\grqq-Zähler ($NG$-Counter), welche für den Ablauf des Space Colonization Algorithmus benötigt werden.	
\end{description}

Ein Objekt der Datenklasse kann als Astsegment eines biologischen Baumes angesehen werden und wird durch das Modellgenerierungssystem als solches visualisiert. Im Folgenden wird der Begriff \glqq Astsegment\grqq{} verwendet, um ein Objekt der Datenklasse der Baumrepräsentation zu bezeichnen.

\section{L-Systeme} \label{sec:ImplementierungLS}

Die Implementierung der Funktionsweise von L-Systemen wird durch einen Unreal-Actor verwirklicht, der im Level platziert werden kann. Nach Start des Levels wird das angegebene Axiom anhand der Produktionsregeln abgeleitet und die sich ergebende Zeichenkette von der Turtle-Implementierung interpretiert.

\subsection{Parameter}

Dem L-System-Actor werden die folgenden Parameter über die Editor-UI übergeben:

\begin{description}
	\item \textbf{Anzahl der Ableitungen:} Die Anzahl der Ableitungen in $\mathbb{N}^+$, welche auf dem Axiom durchgeführt werden. \\
	
	\item \textbf{Axiom:} Das Axiom in Form einer Zeichenkette. \\
	
	\item \textbf{Konstanten:} Eine Konstante besteht aus der Angabe eines Identifikationssymbols und eines Wertes in $\mathbb{R}$. Konstanten können im Axiom und in den Nachfolgern der Produktionen verwendet werden.\\	
	
	\item \textbf{Produktionen:} Jede Produktion besteht aus Angabe eines Vorgängers, eine Liste von Parametern und einem Nachfolger. Der Vorgänger und jeder Parameter entspricht einem einzelnen Symbol, der Nachfolger wird als Zeichenkette eingetragen. Die Parametersymbole können nur innerhalb des Nachfolgers verwendet werden. Der Vorgänger und die Liste der Parameter bilden das parametrische Wort, welches bei einer Ableitung durch den Nachfolger ersetzt wird. \\
	
	\item \textbf{Tropismus:} Der Einfluss von Tropismus in Form eines dreidimensionalen Vektors und eines Biegsamkeitsfaktors.
\end{description}
\begin{figure} [hbtp]
	\centering
	\includegraphics[height=0.4\textheight]{images/LS_ExampleUE4UI.png}
	\caption{Ein Beispiel für die Angabe des L-Systems aus Gleichung \ref{eq:ProdBranching2} mit der resultierenden Baumstruktur aus Abbildung \ref{fig:Branching2L15D25}.}
	\label{fig:LS_ExampleUE4UI}
\end{figure}
Für das Axiom und die Produktionen gelten die in Kapitel \ref{ch:LSysteme} festgelegten Regeln für die Definition von L-Systemen. Weiterhin gelten die Regeln für die Angabe von arithmetischen Operationen. Die Verwendung von Klammern ist jedoch auf die Angabe von Parametern eines parametrischen Wortes beschränkt, ihre Verwendung zur Beeinflussung der Auswertungsreihenfolge eines arithmetischen Ausdrucks wird nicht unterstützt.

Ein Beispiel für die korrekte Eingabe eines L-Systems über die Editor-UI wird in Abbildung \ref{fig:LS_ExampleUE4UI} gezeigt.

\subsection{Ableitung}

Zu Anfang der Erstellung des L-System-Actors werden alle Konstantensymbole im Axiom und den Produktionen durch die Konstantenwerte ersetzt. 

Die Implementierung arbeitet durchgehend auf derselben Zeichenkette, angefangen mit dem Axiom. In jeder Ableitung werden die in den Produktionsregeln definierten parametrischen Wörter durch die angegebenen Nachfolger ersetzt.

Nachdem die vorgegebene Anzahl von Ableitungen durchgeführt wurde, wird die resultierende Zeichenkette an die Turtle-Implementierung weiter geleitet.

\subsection{Turtle Interpretation} \label{subsec:TurtleInterpretationImplementation}

Der Turtle-Implementierung wird die aus den Ableitungen resultierende Zeichenkette übergeben. Diese wird daraufhin sequentiell abgearbeitet und entsprechend den in Abschnitt \ref{sec:LS_Baumstrukturen} vorgestellten Konzepten interpretiert. Der resultierende Baum wird für die Konstruktion des Modells an das Modellgenerierungssystem weitergegeben.

\section{Space Colonization Algorithmus} \label{sec:ImplementierungSCA}

Die Implementierung einer Space-Colonization-Algorithmus-Baumstruktur stellt sich aus der Platzierung von mindestens einem Actor für die Repräsentation des Einflussbereichs und einem Actor für die Generierungsprozedur zusammen.

\subsection{Einflussbereiche}

Durch die Platzierung von Einflussbereich-Actors kann die Verteilung von Einflusspunkten mithilfe des Unreal-Editors angepasst werden. Es sind derzeit zwei Formen von Einflussbereichen wählbar: Eine Kugel- und eine Zylinderform. Die Kugelform erfordert die Angabe eines Kugelradius während die Zylinderform durch Höhe und Radius beschrieben wird. 

Einem Space-Colonization-Actor können mehrere Einflussbereiche zugeordnet werden, um eine bestimmte Baumstruktur zu formen. Jeder Einflussbereich-Actor wird weiterhin in einem vorgegebenen Abstand zum Space-Colonization-Actor platziert.

Dem Einflussbereich muss eine positive Anzahl von zu generierenden Einflusspunkten und ein Random-Seed Wert übergeben werden. Ein Random-Seed ist ein Wert, der von einem Zufallsgenerator verwendet wird, um eine zufällig wirkende Folge von Zahlen zu generieren. Bei Verwendung desselben Random-Seeds wird dieselbe Folge von Zufallszahlen erstellt, was eine gewisse Kontrolle über die Generierung ermöglicht. Somit wird, wenn alle anderen Parameter ebenfalls übereinstimmen, mit demselben Random-Seed Wert dieselbe Baumstruktur aufgebaut. 

\subsection{Parameter}

Dem Space-Colonization-Actor werden Parameter des ursprünglichen Algorithmus sowie Eingaben für die in Abschnitt \ref{sec:SCA_Erweiterungen} besprochenen Erweiterungen über das Editor-UI übergeben. Zu den ursprünglichen Parametern gehören der Minimalradius, Einflussradius, Schrittweite, ein Tropismusvektor sowie die Anzahl der durchzuführenden Iterationen. Zu den erweiterten Parametern gehören der maximale Grad, die maximale Zweigtiefe, die maximale Anzahl von \glqq Kein Wachstum\grqq{}-Iterationen und eine Abfrage, ob gewichtetes Wachstum durchgeführt werden soll.

Weiterhin müssen die zugeordneten Einflussbereich-Actors angegeben werden -- damit der Algorithmus durchgeführt werden kann, muss mindestens einer dieser Actors mit mindestens einem Einflusspunkt eingetragen werden.

\subsection{Ablauf des Algorithmus}

Die Einflusspunkte aller dem Space-Colonization-Actor zugeordneten Einflussbereiche werden diesem zu Beginn der Baum-Generierung übergeben. Daraufhin wird ein Baum entsprechend der in Abschnitt \ref{sec:GenerierungBaumstrukturen} und Abschnitt \ref{sec:SCA_Erweiterungen} vorgestellten Konzepte aufgebaut und für die Konstruktion des Modells an das Modellgenerierungssystem weitergegeben.

\section{Modellgenerierung} \label{sec:Modellgenerierung}

L-System- und Space-Colonization-Actors übergeben dem Modellgenerierungssystem nach Ablauf der Generierung einen graphentheoretischen Baum. Das Modellgeneriungssystem erstellt daraus ein dreidimensionales Mesh (engl. für Polygonnetz) in der vom Framework geforderten Form. Das Mesh entspricht der Visualisierung der Kanten des graphentheoretischen Baums in Form von Zylindern und simuliert dadurch vereinfacht die Aststruktur eines biologischen Baumes. \cite[S.2]{SpaceColonizationAlgorithm:07} 

\subsection{Procedural Mesh Component}

Die Procedural Mesh Component ist eine Komponente der Unreal Engine, welche die Darstellung von prozedural generierten Polygonnetzen zulässt. Ein Vertex ist ein Punkt in einem Polygonnetz mit zur Visualisierung benötigten Informationen. Der Komponente werden Vertexdaten in Form von Listen aus Positions-, Normalen-, Tangenten-, Textur- und Indexdaten übergeben. Das Grafiksystem der Unreal Engine ist daraufhin in der Lage, die Komponente als dreidimensionales Modell im Level darzustellen. \cite{ProceduralMeshComponent:15} Die Vertexdaten werden, basierend auf dem übergebenen Baum und den Parametern, vom Modellgenerierungssytem erstellt.

Jedem L-System-Actor und Space-Colonization-Actor ist eine Procedural Mesh Component zugeordnet.

\subsection{Parameter} \label{subsec:Modellgenerierung_Parameter}

Dem Modellgenerierungssystem werden die folgenden Parameter über das Editor-UI übergeben:

\begin{description}
	\item \textbf{Radius-Daten:} Diese beinhalten den Blattradius, den Radiuswachstumswert, den Stammbreitenmultiplikator und die Abfrage, ob Radiusberechnungen durchgeführt werden sollen. \\
	
	\item \textbf{Genauigkeit:} Diese beinhalten die minimale und maximale Anzahl von Zylindersektionen sowie den Kurvenreduktionswert. \\
	
	\item \textbf{Sonstiges:} Weiterhin wird ein Startrotationswinkel, ein Material und eine Abfrage, ob ein fraktales Mesh erstellt werden soll, übergeben. Ein Material beinhaltet Textur- und Shaderinformationen und ist für die Oberflächenbeschaffenheit des generierten Modells verantwortlich.
\end{description}

\subsection{Operationen auf dem Baum}

Folgende Operationen werden vor Beginn der Modelldatengenerierung auf dem graphentheoretischen Baum durchgeführt:

\begin{description}
	\item \textbf{Kurvenreduktion:} Die Kurvenreduktion wird, beginnend mit dem Wurzel-Astsegment des Baums, rekursiv ausgeführt. In jedem Schritt wird überprüft ob das aktuelle Astsegment entsprechend der Beschreibung in Paragraph \ref{par:Kurvenreduktion} entfernt werden kann. Falls nicht, wird die Kurvenreduktion auf allen Nachfolgern des aktuellen Astsegment durchgeführt. Die Rekursion bricht ab, falls ein Astsegment keine Nachfolger besitzt.
	
	Der Parameter \glqq Kurvenreduktionswert\grqq{} entspricht dem Maximalwert des Skalarprodukts $max_K$.\\
	
	\item \textbf{Radiusberechnung:} Der Endradius jedes Astsegments wird, beginnend mit dem Wurzel-Astsegment des Baums, rekursiv anhand von Gleichung \ref{eq:Radiusberechnung} berechnet. Der Parameter \glqq Blattradius\grqq{} entspricht $r_0$ und der \glqq Radiuswachstumswert\grqq{} entspricht $g$. 
	
	Der Startradius jedes Astsegments wird auf den Wert des Endradius seines Vorgängers gesetzt. Da das Wurzel-Astsegment keinen Vorgänger besitzt, wird der Startradius aus der Multiplikation des Stammbreitenmultiplikators mit dem Endradius des Objekts bestimmt.\\
	
	\item \textbf{Normalenberechnung:} Normalen werden für die Berechnung der Modelldaten benötigt. Die Endnormale $\overrightarrow{n_{e}}$ jedes Astsegments wird mithilfe der Startposition $\overrightarrow{p_{s}}$ und Endposition $\overrightarrow{p_{e}}$ sowie seiner Startnormale $\overrightarrow{n_{s}}$ wie folgt berechnet:
	\begin{equation}
		\overrightarrow{n_{e}} = \dfrac{(\overrightarrow{p_{e}} - \overrightarrow{p_{s}}) + \overrightarrow{n_{s}}}{\lVert (\overrightarrow{p_{e}} - \overrightarrow{p_{s}}) + \overrightarrow{n_{s}} \rVert}
	\end{equation}
	Die Startnormale entspricht der Endnormale des Vorgängers, im Falle des Wurzel-Astsegments entspricht die Startnormale $\overrightarrow{n_s} = \overrightarrow{p_{e}} - \overrightarrow{p_{s}}$. \\
	
	
	\item \textbf{Verringerung der Abzweigungswinkel:} Die Nachfolger jedes Astsegments werden, entsprechend der Beschreibung in Paragraph \ref{par:VerringerungAbzweigungswinkel}, einander angenähert. Die Verringerung der Abzweigungswinkel wird nur bei durch Space-Colonization-Actors generierten Bäumen durchgeführt.
\end{description}


\subsection{Generierung der Zylinder-Meshes} \label{subsec:ZylinderMeshes}

Moderne Grafik-APIs stellen dreidimensionale Modelle in Form von geometrischen Primitiven dar. Dem Unreal-Engine-Grafiksystem werden Primitive in Form von Dreiecken übergeben. Je drei Vertizes bilden ein Dreieck, dessen Oberfläche mithilfe eines übergebenen Materials gefärbt wird.

Die Vertizes bestehen zusätzlich zu ihren Positionen aus :

\begin{description}
	\item \textbf{Normalenvektoren} Stellen die Richtung dar, von der aus das Dreieck sichtbar ist und werden unter anderem für Beleuchtungsberechnungen benötigt. \cite{ModelingByNumbers1A:13} 
	\item \textbf{Tangentenvektoren} Stehen orthogonal zu den Normalenvektoren und werden unter anderem für erweiterte Beleuchtungsberechnungen verwendet. \cite{ModelingByNumbers1A:13} 
\end{description}

Mithilfe dieser Informationen werden zusätzlich Texturkoordinaten, welche für das korrekte Auftragen von Texturen auf das Modell benötigt werden, und Indexdaten, welche für die Darstellung des Polygonnetzes benötigt werden, berechnet.

Der Mantel eines Zylinders kann nun durch die Verbindung von zwei Kreisen generiert werden. 

\paragraph{Berechnung der Vertizes auf einem Kreis}

Da ein Kreis theoretisch aus unendlich vielen Punkten besteht, muss eine gewisse Genauigkeit bei der Darstellung von runden Modellen festgelegt werden -- ein Kreis wird aus einer zuvor definierten Anzahl von Segmenten generiert. Mithilfe eines Mittelpunkts $\overrightarrow{c}$ und eines Radius $r$ kann ein Kreis im zweidimensionalen Raum beschrieben werden. Die Vertexpositionen $\overrightarrow{v_i}$ werden wie folgt berechnet:

\begin{equation}
	\overrightarrow{v_i} =\overrightarrow{c} + r * \begin{pmatrix}
	cos(d)\\
	sin(d)\\
	0
	\end{pmatrix}
	\text{ mit } d = rot_z + i * \frac{360\degree}{n} \text{ und } i = 0 ... (n-1)
\end{equation}
\cite{ModelingByNumbersZylindersA:13}

wobei $n$ der Anzahl der Segmente und $rot_z$ einem Startrotationswinkel entspricht.

Mithilfe der Kreisnormalen $\overrightarrow{n_k}$, die orthogonal zur Kreisebene steht und des Kreismittelpunkts $\overrightarrow{c}$ kann die Kreisebene beschrieben werden, auf welcher die Vertizes zu generieren sind. Eine Vertexposition $\overrightarrow{v_i}$ muss um den Kreismittelpunkt rotiert werden, um auf der Kreisebene zu liegen, wobei Rotationsachse $\overrightarrow{R}$ und Rotationswinkel $\alpha$ wie folgt berechnet werden:
\begin{equation}
\overrightarrow{R} = \dfrac{(\overrightarrow{v_i} - \overrightarrow{c}) \times \overrightarrow{n_k}}{\lVert (\overrightarrow{v_i} - \overrightarrow{c}) \times \overrightarrow{n_k} \rVert}
\end{equation}

\begin{equation}
\alpha = arccos(\langle \overrightarrow{n_k}, \overrightarrow{z} \rangle) \text{ mit } \overrightarrow{z} = \begin{pmatrix}
0\\
0\\
1
\end{pmatrix}
\end{equation}
 wobei $arccos$ dem Arkuskosinus entspricht. \cite{RotationBetweenVectors:16} 
 Die rotierte Vertexposition $\overrightarrow{v}$ ermöglicht die Berechnung der Vertexnormalen $\overrightarrow{n_v}$ und Vertextangente $\overrightarrow{t_v}$:
 
 \begin{equation}
	 \overrightarrow{n_v} = \dfrac{\overrightarrow{v} - \overrightarrow{c}}{\lVert \overrightarrow{v} - \overrightarrow{c} \rVert} \text{ und } \overrightarrow{t_v} = \overrightarrow{n_v} \times \overrightarrow{n_k}
 \end{equation}
\begin{figure} [hbtp]
	\centering
	\includegraphics[height=0.25\textheight]{images/Ring6Sections.png}
	\caption{Beispiel für die Berechnung von Vertexpositionen auf einem Einheitskreis mit einer Genauigkeit von sechs Segmenten. $d = \frac{360\degree}{6} = 60 \degree$. Die blauen Punkte entsprechen den Positionen, die grünen Pfeile den Normalen und die roten Pfeile den Tangenten der Vertizes. Eigene Abbildung.}
	\label{fig:Ring6Sections}
\end{figure}
\paragraph{Verbindung der Kreise}
Um den Zylindermantel eines Astsegments zu bilden, werden zwei Kreise mithilfe der Start- und Enddaten des Objekts generiert. Jedes Segment eines Kreises wird mit dem entsprechenden Segment des anderen Kreises verbunden und bildet dadurch ein Zylindersegment. Ein Zylindersegment entspricht somit einem Rechteck. Da das Modell jedoch Dreiecksdaten benötigt, wird jedes Rechteck, wie in Abbildung \ref{subfig:Zylinder10SegmenteWireframe} dargestellt, aus zwei Dreiecken gebildet. \cite{ModelingByNumbersZylindersA:13}
\begin{figure} [hbtp]
\centering
\begin{subfigure}[t]{.4\textwidth}
	\centering
	\includegraphics[height=.75\linewidth]{images/Zylinder10SegmenteWireframe.png}
	\caption{Darstellung der Dreiecke, welche bei der Verbindung der Kreise entstehen.}
	\label{subfig:Zylinder10SegmenteWireframe}
\end{subfigure}
\hspace{.1\textwidth}
\begin{subfigure}[t]{.4\textwidth}
	\centering
	\includegraphics[height=.75\linewidth]{images/Zylinder10SegmenteOpaque.png}
	\caption{Gefärbte Dreiecke mit Beleuchtungsberechnung.}
	\label{subfig:Zylinder10SegmenteOpaque}
\end{subfigure}
\caption{Verbindung zweier Kreise zu einem Zylindermantel. Eigene Abbildungen.}
\label{fig:Zylinder10Segmente}
\end{figure}

Wird jedes Astsegment durch die Verbindung von genau zwei Kreisen dargestellt, führt dies zu der Generierung redundanter Vertexdaten. Die Enddaten eines Astsegments und die Startdaten seines Nachfolgers entsprechen einander, da die daraus generierten Kreis-Vertizes genau aufeinander liegen. Anstatt nun vier Kreise für die Generierung zweier Zylinder zu verwenden, können die Vertizes des verbindenden Kreises wiederverwendet werden -- somit sind drei Kreise für die Generierung zweier Zylinder ausreichend.

Die Verbindung der Modelldaten kann für alle Astsegmente durchgeführt werden, deren Nachfolger dieselbe Zweigtiefe besitzen und somit eine Folge von zusammenhängenden Zylindermodellen bilden. Für einen Nachfolger mit einer sich unterscheidenden Zweigtiefe wird eine neue Folge von zusammenhängenden Zylindermodellen begonnen. \cite{ModelingByNumbersZylindersA:13}

Ein Beispiel für die Generierung zweier Zylinder mithilfe von drei Kreisen wird in Abbildung \ref{subfig:MultiZylinder10SegmenteWireframe} dargestellt.

\begin{figure} [hbtp]
	\centering
	\begin{subfigure}[t]{.4\textwidth}
		\centering
		\includegraphics[height=\linewidth]{images/MultiZylinder10SegmenteWireframe.png}
		\caption{Darstellung der Dreiecke, welche bei der Verbindung der Kreise entstehen.}
		\label{subfig:MultiZylinder10SegmenteWireframe}
	\end{subfigure}
	\hspace{.1\textwidth}
	\begin{subfigure}[t]{.4\textwidth}
		\centering
		\includegraphics[height=\linewidth]{images/MultiZylinder10SegmenteOpaque.png}
		\caption{Gefärbte Dreiecke mit Beleuchtungsberechnung.}
		\label{subfig:MultiZylinder10SegmenteOpaque}
	\end{subfigure}
	\caption{Verbindung dreier Kreise zu einer Folge zusammenhängender Zylindermodelle. Eigene Abbildungen.}
	\label{fig:MultiZylinder10Segmente}
\end{figure}





\chapter{Ergebnisse}
In diesem Kapitel werden die Ergebnisse vorgestellt, die mithilfe den Implementierungen von L-Systemen und dem Space-Colonization Algorithmus produziert werden können.

Bei der Erstellung der Abbildungen wurden die Beleuchtungsberechnungen des Grafiksystems der Unreal Engine aktiviert.
\section{L-System-Akteur}
L-Systeme ermöglichen es mithilfe bestimmter Produktionsregeln Baumstrukturen zu generieren. Um Regeln zu finden, die realitätsnahe Ergebnisse liefern kann das Abzweigungsverhalten von biologischen Bäumen betrachtet werden. 
\subsection{Monopodial}
Ein biologischer Baum mit monopodialem Abzweigungsverhalten bildet einen Hauptstamm, der stets weiterwächst, mit davon abzweigenden Nebenästen. Ein Beispiel für ein solches Wachstum wäre das folgende L-System:

\begin{equation}
\begin{array}{llll}
\omega & : A(100) \\
p_1 & : A(l) &\rightarrow& F(l)[\&(a1)B(l*r2)]/(d)A(l*r1) \\
p_2 &  : B(l) &\rightarrow& F(l)[-(a2)B(l*r2)]/(b1)B(l*r2)\backslash(b2)
\end{array}
\label{eq:ProdMonopodial}
\end{equation} 
\subsection{Simpodial}
test
\subsection{Ternäre Verzweigungen}
test
\subsection{Tropismus}
Die Wahl eines Tropismusvektors erweitert diese Regeln und ermöglicht
\subsection{Performanz}
test
\section{Space-Colonization-Akteur}
test
\subsection{Ursprüngliche Parameter}
test
\subsection{Erweiterte Parameter}
test

\subsection{Probleme}
Mit SCA: Gleicher Abstand von zwei Einflusspunkten führt zu beinahe unendlicher Hin- und Her-Generierung von Branches.
\\
SCA: Random-Verteilung der Punkte.

\section{Performanz}



\chapter{Zusammenfassung und Ausblick}

Zusammenfassung

\section{Erweiterungen}
test
\subsection{Space Colonization Algorithmus}
test
\paragraph{Positionsabfragen}
Verbesserte Positionsvergleiche bei Einflusspunkt- zu Knotenpunkt-Abfragen.

\paragraph{Einflussbereiche} 
Verbesserte Möglichkeit, Einflussbereiche anzugeben. Oberfläche, Random-Verteilung

\subsection{Aststruktur}

Generalized cylinders.

\subsection{Texturen}
test
\subsection{Blätter}
test
\subsection{Generierung zur Laufzeit}
test
\subsection{Verteilung}
test
\subsection{Instanziierung}
Eine beschränkte Anzahl Pflanzen generieren und diese so verteilen, dass es nicht bemerkt wird, dass es immer die selben Pflanzen sind

\subsection{title}
test
%
% ...
%--------------------------------------------------------------------------
\backmatter                        		% Anhang
%-------------------------------------------------------------------------
\bibliographystyle{geralpha}			% Literaturverzeichnis
\bibliography{literatur}     			% BibTeX-File literatur.bib
%--------------------------------------------------------------------------
\printindex 							% Index (optional)
%--------------------------------------------------------------------------
\begin{appendix}						% Anhänge sind i.d.R. optional
   \chapter{Glossar}
\abbreviation{Mesh}			{Englisch für Polygonnetz.}			% Glossar   
   \chapter{Erklärung der Kandidatin / des Kandidaten}

\begin{description}[$\Box$~]
\item[$\Box$] Die Arbeit habe ich selbstständig verfasst und keine anderen als die angegebenen Quellen- und Hilfsmittel verwendet.\\

\item[$\Box$] Die Arbeit wurde als Gruppenarbeit angefertigt. Meine eigene Leistung ist\\
...\\

Diesen Teil habe ich selbstständig verfasst und keine anderen als die angegebenen Quellen und Hilfsmittel verwendet. \\

Namen der Mitverfasser: ...

\end{description}

\vspace{2cm}

\begin{minipage}[t]{3cm}
\rule{3cm}{0.5pt}
Datum
\end{minipage}
\hfill
\begin{minipage}[t]{9cm}
\rule{9cm}{0.5pt}
Unterschrift der Kandidatin / des Kandidaten
\end{minipage}	% Selbstständigkeitserklärung
\end{appendix}

\end{document}
