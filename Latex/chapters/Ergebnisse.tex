\chapter{Ergebnisse}
In diesem Kapitel werden die Ergebnisse vorgestellt, die mithilfe den Implementierungen von L-Systemen und dem Space-Colonization Algorithmus produziert werden können.

Bei der Erstellung der Abbildungen wurden die Beleuchtungsberechnungen des Grafiksystems der Unreal Engine aktiviert.
\section{L-System-Akteur}
L-Systeme ermöglichen es mithilfe bestimmter Produktionsregeln Baumstrukturen zu generieren. Um Regeln zu finden, die realitätsnahe Ergebnisse liefern kann das Abzweigungsverhalten von biologischen Bäumen betrachtet werden. 
\subsection{Monopodial}
Ein biologischer Baum mit monopodialem Abzweigungsverhalten bildet einen Hauptstamm, der stets weiterwächst, mit davon abzweigenden Nebenästen. Ein Beispiel für ein solches Wachstum wäre das folgende L-System:

\begin{equation}
\begin{array}{llll}
\omega & : A(100) \\
p_1 & : A(l) &\rightarrow& F(l)[\&(a1)B(l*r2)]/(d)A(l*r1) \\
p_2 &  : B(l) &\rightarrow& F(l)[-(a2)B(l*r2)]/(b1)B(l*r2)\backslash(b2)
\end{array}
\label{eq:ProdMonopodial}
\end{equation} 
\subsection{Simpodial}
test
\subsection{Ternäre Verzweigungen}
test
\subsection{Tropismus}
Die Wahl eines Tropismusvektors erweitert diese Regeln und ermöglicht
\subsection{Performanz}
test
\section{Space-Colonization-Akteur}
test
\subsection{Ursprüngliche Parameter}
test
\subsection{Erweiterte Parameter}
test

\subsection{Probleme}
Mit SCA: Gleicher Abstand von zwei Einflusspunkten führt zu beinahe unendlicher Hin- und Her-Generierung von Branches.
\\
SCA: Random-Verteilung der Punkte.

\section{Performanz}


