\chapter{Space Colonization Algorithmus}

TODO: Zusammenfassung Space Colonization Algorithm.

\section{Ursprung}
Der Space Colonization Algorithmus wurde ursprünglich zur Modellierung und Visualisierung von Blattvenen vorgeschlagen und basiert auf der Wirkung des Pflanzenhormons Auxin. Dieser Hormonstoff entsteht im Blatt und wird von bereits existierenden Blattvenen angezogen, der resultierende Hormonstrom führt zur Bildung von neuen Venen im Blatt. Die Simulation dieses Vorgangs führt zu realitätsnahen Venenmustern. \cite[Abschn. 2.5]{LeafVenation:05}

Mithilfe einer Erweiterung in den dreidimensionalen Raum und Nachbearbeitung der Resultate kann eine Vielfalt von Baum- und Strauchstrukturen generiert werden. \cite[Abschn. 1]{SpaceColonizationAlgorithm:07}

\section{Aufbau}

Der Algorithmus verarbeitet eine Menge von Einflusspunkten $S$ und baut darauf basierend einen gerichteten Baum-Graphen $G = \langle V,E\rangle$ auf. Folgende Begriffe werden in diesem Zusammenhang verwendet: 

\begin{description}[labelindent]
	\item[\boldmath$V$] Die Menge der Knotenpunkte. Jeder Knotenpunkt entspricht einem Punkt $\overrightarrow{p_v}$ im dreidimensionalen Raum. \cite[S.358]{ThI:14}\\
	
	\item[\boldmath$E$] Die Menge der Kanten, welche die Vorgänger-Nachfolger-Beziehungen zwischen den Knotenpunkten darstellt. Eine Kante $e \in E$ wird als Tupel $e = (v_1, v_2)$ mit $v_1, v_2 \in V$ dargestellt. Jeder Knoten besitzt maximal einen Vorgänger und eine endliche Menge von Nachfolgern.\cite[S.358]{ThI:14}\\
	
	\item[\boldmath$Wurzel$] Eine Wurzel ist ein Knoten $v \in V$ ohne einen Vorgänger. Der Baum besitzt genau eine Wurzel. \cite[S.358]{ThI:14}\\
	
	\item[\boldmath$Grad$] Der Grad eines Knotens ist definiert als die Anzahl seiner Nachfolger. \cite[S.29]{AlgoDat:14}\\
	
	\item[\boldmath$Tiefe$] Die Tiefe eines Knotens ist definiert als die Länge der Folge von Vorgängern, die durchlaufen werden muss, bis die Wurzel erreicht wurde. Die Wurzel besitzt die Tiefe $0$. \cite[S.30]{AlgoDat:14}

	
\end{description}
Der Algorithmus benötigt die folgenden Eingaben:

\begin{description}[labelindent]
	\item[\boldmath$d_i$] Der Einflussradius. Einflusspunkte prägen den Aufbau des Baums nur, wenn sich Knotenpunkte innerhalb dieses Radius befinden. \cite[Abschn. 2]{SpaceColonizationAlgorithm:07}\\
	
	\item[\boldmath$d_k$] Der Minimalradius. Befindet sich ein Knotenpunkt innerhalb des Minimalradius um einen Einflusspunkt, wird dieser aus der Menge der Einflusspunkte $S$ entfernt. \cite[Abschn. 2]{SpaceColonizationAlgorithm:07}\\
	
	\item[\boldmath$D$] Die Schrittweite. Jeder neu generierte Knotenpunkt wird in diesem Abstand zu seinem Vorgänger positioniert. \cite[Abschn. 2]{SpaceColonizationAlgorithm:07} \\
	
	\item[\boldmath$\overrightarrow{T}$] Der Tropismusvektor. Tropismus ist die Tendenz einer Pflanze in eine bestimmte Richtung zu wachsen, beispielsweise aufgrund einer Lichtquelle oder der Beugung durch Gravitation.\cite[Abschn. 3]{SpaceColonizationAlgorithm:07} 

\end{description}

\section{Ablauf}
\label{sec:SCA_Ablauf}
Zu Beginn des Algorithmus werden $N$ Einflusspunkte in einem vorgegebenen Bereich generiert. Dieser Einflussbereich signalisiert die Verfügbarkeit von Raum, in dem der Baum wachsen kann.  \cite[Abschn. 2]{SpaceColonizationAlgorithm:07}

Daraufhin wird der Baum iterativ aufgebaut und durchläuft in jeder Iteration die folgenden Schritte: 

\begin{description}[labelindent]
	\item[\boldmath$1.$] Für jeden Einflusspunkt in $S$ wird der am nächsten liegende Knotenpunkt $v\in V$ bestimmt. Befindet sich $v$ innerhalb des Einflussradius $d_i$ um den Einflusspunkt herum, wird dieser einer zugeordneten Menge $S(v)$ hinzugefügt. $S(v)$ beinhaltet somit alle Einflusspunkte, die einen Einfluss auf den Knotenpunkt ausüben. \cite[Abschn. 2]{SpaceColonizationAlgorithm:07} \label{alg:SCA_1}\\
	
	\item[\boldmath$2.$] Befinden sich Elemente in $S(v)$, wird ein neuer Knotenpunkt $v'$ den Nachfolgern von $v$ hinzugefügt und $v$ als Vorgänger von $v'$ eingetragen.  Alle Punkte in $S(v)$ beeinflussen $v'$ in gleichem Maße, die neue Position $\overrightarrow{p_{v'}}$ des Knotenpunkts berechnet sich somit wie folgt:
	
	\begin{equation}
	\begin{array}{ll}
	\overrightarrow{p_{v'}} & = \overrightarrow{p_v} + D * \tilde{n}
	\end{array}
	\end{equation} 
	
	\begin{equation}
	\begin{array}{ll}
	\text{  mit  } \tilde{n} & = \dfrac{\overrightarrow{n} + \overrightarrow{T} }{\lVert\overrightarrow{n} + \overrightarrow{T}\rVert}  
	\end{array}
	\end{equation} 
	
	\begin{equation}
	\begin{array}{ll}
	\text{ und }  \overrightarrow{n} = \sum\limits_{s \in S(v)}\dfrac{\overrightarrow{p_s} - \overrightarrow{p_v}}{\lVert \overrightarrow{p_s} - \overrightarrow{p_v} \rVert}
	\end{array}
	\end{equation}	
	\cite[Abschn. 2]{SpaceColonizationAlgorithm:07} \label{alg:SCA_2}\\
	
	\item[\boldmath$3.$] Für jeden Einflusspunkt wird überprüft, ob sich ein Knotenpunkt innerhalb des Minimalradius $d_k$ befindet. Existiert ein solcher Knotenpunkt, wird der Einflusspunkt aus der Menge der Einflusspunkte $S$ entfernt. \cite[Abschn. 2]{SpaceColonizationAlgorithm:07} \label{alg:SCA_3}
\end{description}

Diese Schritte werden solange ausgeführt, bis alle Einflusspunkte entfernt wurden, sich kein Knotenpunkt im Einflussradius eines Einflusspunktes befindet oder bis eine vorgegebene Maximalanzahl von Iterationen durchgeführt wurde. Abbildung \ref{fig:SCA_Basic} zeigt eine beispielhafte Anwendung des Algorithmus.

\begin{figure} [hbtp]
	\centering
	\begin{subfigure}[t]{.3\textwidth}
		\centering
		\includegraphics[width=\linewidth]{images/SCA_Basic1.png}
		\caption{ Ausgangssituation. }
		\label{subfig:SCA_Basic1}
	\end{subfigure}
	\hspace{.03\textwidth}
	\begin{subfigure}[t]{.3\textwidth}
		\centering
		\includegraphics[width=\linewidth]{images/SCA_Basic2.png}
		\caption{Schritt 1: Blaue Kreise entsprechen dem Einflussradius.}
		\label{subfig:SCA_Basic2}
	\end{subfigure}
	\hspace{.03\textwidth}
	\begin{subfigure}[t]{.3\textwidth}
		\centering
		\includegraphics[width=\linewidth]{images/SCA_Basic3.png}
		\caption{Schritt 2: Bestimmung von $\tilde{n}$ (roter Pfeil) ohne Einbeziehung von $\overrightarrow{T}$.}
		\label{subfig:SCA_Basic3}
	\end{subfigure}

	\begin{subfigure}[t]{.3\textwidth}
		\centering
		\includegraphics[width=\linewidth]{images/SCA_Basic4.png}
		\caption{Schritt 2: Platzierung des neuen Knotenpunkts.}
		\label{subfig:SCA_Basic4}
	\end{subfigure}
	\hspace{.03\textwidth}
	\begin{subfigure}[t]{.3\textwidth}
		\centering
		\includegraphics[width=\linewidth]{images/SCA_Basic5.png}
		\caption{Schritt 3: Rote Kreise entsprechen dem Minimalradius.}
		\label{subfig:SCA_Basic5}
	\end{subfigure}
	\hspace{.03\textwidth}
	\begin{subfigure}[t]{.3\textwidth}
		\centering
		\includegraphics[width=\linewidth]{images/SCA_Basic6.png}
		\caption{Ausgangssituation der nächsten Iteration.}
		\label{subfig:SCA_Basic6}
	\end{subfigure}

	\begin{subfigure}[t]{.3\textwidth}
		\centering
		\includegraphics[width=\linewidth]{images/SCA_Basic7.png}
		\caption{Einflusspunkte beeinflussen unterschiedliche Knotenpunkte.}
		\label{subfig:SCA_Basic7}
	\end{subfigure}
	\hspace{.03\textwidth}
	\begin{subfigure}[t]{.3\textwidth}
		\centering
		\includegraphics[width=\linewidth]{images/SCA_Basic8.png}
		\caption{Eine Verzweigung entsteht.}
		\label{subfig:SCA_Basic8}
	\end{subfigure}
	\caption{Beispielhafte Anwendung des Space Colonization Algorithmus. Blaue Punkte entsprechen Einflusspunkten, weiße Punkte entsprechen Knotenpunkten. Der unterste Knotenpunkt stellt die Wurzel dar. Eigene Abbildungen auf Grundlage von \cite[Abb. 2]{SpaceColonizationAlgorithm:07}.}\label{fig:SCA_Basic}
	
\end{figure}


\section{Modellierung von Baumstrukturen}
\label{sec:ModellierungBaumstrukturen}
Der Space Colonization Algorithmus liefert einen Baum-Graphen, der Knotenpunkte enthält, welche Positionen im dreidimensionalen Raum darstellen. Um diese Knotenpunkte in Form von baumähnlichen Strukturen zu visualisieren, wird die Prozedur erweitert. Die Modellierung der Baumstrukturen läuft wie folgt ab:

\begin{description}[labelindent]
	\item[\boldmath$1.$] Der Einflussbereich wird mit der vorgegeben Anzahl von Einflusspunkten gefüllt. \cite[Abschn. 2]{SpaceColonizationAlgorithm:07} \\

	\item[\boldmath$2.$] Der Baum-Graph wird wie in Abschnitt \ref{sec:SCA_Ablauf} beschrieben iterativ generiert. \cite[Abschn. 2]{SpaceColonizationAlgorithm:07} \\

	\item[\boldmath$3.$] Die Nachfolger jedes Knotenpunkts werden einander angenähert, um eine Verringerung der Abzweigungswinkel zwischen den verbindenden Kanten zu erreichen. Dies führt zu einer insgesamt realistischeren Baumstruktur. \cite[Abschn. 2]{SpaceColonizationAlgorithm:07} \\
	
	\item[\boldmath$4.$] Die Kanten, welche die Knotenpunkte verbinden, werden mithilfe von Zylindern visualisiert, um die Aststruktur eines Baumes zu simulieren. \cite[Abschn. 2]{SpaceColonizationAlgorithm:07} 
	
\end{description}

Die Berechnung der Zylinderbreiten erfolgt anhand von Murrays Regel, die besagt, dass der Radius $r$ eines Astes auf Grundlage der Radien $r_{n_1}...r_{n_m}$ der nachfolgenden, von ihm abzweigenden Äste wie folgt berechnet werden kann: 

\begin{equation}
\begin{array}{ll}
r^g & = r_{n_1}^g + r_{n_2}^g + ... + r_{n_m}^g 
\end{array}
\end{equation} 

Diese Berechnung kann rekursiv auf dem Baum-Graphen ausgeführt werden, indem für jeden Knotenpunkt der Radius aus den Radien seiner Nachfolger berechnet wird. Besitzt ein Knotenpunkt keine Nachfolger, wird ein festgelegter Radius $r_0$ zurückgegeben. \cite[Abschn. 3.5]{LeafVenation:05} Der Wert $g$ kann frei gewählt werden.

\begin{figure} [hbtp]
	\centering
	\begin{subfigure}[t]{.35\textwidth}
		\centering
		\includegraphics[width=\linewidth]{images/SCA_Extended1.png}
		\caption{$N=2000$ Einflusspunkte, zufällig in einem Ring mit dem Radius $r = 500$ verteilt.}
		\label{subfig:SCA_Extended1}
	\end{subfigure}
	\hspace{.1\textwidth}
	\begin{subfigure}[t]{.35\textwidth}
		\centering
		\includegraphics[width=\linewidth]{images/SCA_Extended2.png}
		\caption{Ergebnis des Space Colonization Algorithmus mit einem Einflussradius $d_i = 100$, Minimalradius $d_k = 20$, Schritteweite $D = 15$ und $\overrightarrow{T} = \overrightarrow{0}$.}
		\label{subfig:SCA_Extended2}
	\end{subfigure}
	\begin{subfigure}[t]{.35\textwidth}
		\centering
		\includegraphics[width=\linewidth]{images/SCA_Extended3.png}
		\caption{Verringerung der Abzweigungswinkel.}
		\label{subfig:SCA_Extended3}
	\end{subfigure}
	\hspace{.1\textwidth}
	\begin{subfigure}[t]{.35\textwidth}
		\centering
		\includegraphics[width=\linewidth]{images/SCA_Extended4.png}
		\caption{Modellierung des Ergebnis mithilfe von Zylindern mit $r_0 = 1$ und $g=2$.}
		\label{subfig:SCA_Extended4}
	\end{subfigure}
	\caption{Modellierung einer zweidimensionalen Baumstruktur entsprechend der in Abschnitt \ref{sec:ModellierungBaumstrukturen} beschriebenen Schritte. Eigene Abbildungen.}
	\label{fig:SCA_Extended}
\end{figure}

Die von Runions u.a. \cite{SpaceColonizationAlgorithm:07} vorgeschlagene Kurven-Unterteilung \cite[Abschn. 2]{SpaceColonizationAlgorithm:07} wurde in dieser Arbeit nicht behandelt, da durch Angabe der Schrittweite $D$ eine ausreichende visuelle Qualität erzielt werden kann.

Abbildung \ref{fig:SCA_Extended} zeigt die Modellierung einer zweidimensionalen Baumstruktur.

\section{Erweiterungen}

Die Zweigtiefe $Z(v)$ eines Knotens $v \in V$ wird für die Implementierung und Erweiterung des Space Colonization Algorithmus verwendet. Sie entspricht nicht der Tiefe des Knotens und berechnet sich wie folgt:
\begin{equation}
Z(v)= \begin{cases}
0 & \text{falls v die Wurzel ist} \\
Z(v') & \text{falls v' genau einen Nachfolger besitzt}\\
1 + Z(v') & \text{sonst}
\end{cases} 
\end{equation}
wobei $v'$ den Vorgänger des Knotens $v$ darstellt. \\

Um mehr Kontrolle über die vom Space Colonization Algorithmus generierte Baumstruktur zu bekommen, wurden zusätzliche Bedingungen für die Erstellung von neuen Knotenpunkten hinzugefügt:
\begin{description}[labelindent]
	\item[\boldmath$max_{grad}$] Der maximale Grad der Knotenpunkte im Baum-Graphen. Bevor ein neuer Knotenpunkt $v_n$ als Nachfolger eines Knotenpunkts $v$ erstellt wird, wird die Anzahl der Nachfolger von $v$ überprüft. Entspricht diese $max_{grad}$, wird der Knotenpunkt $v_n$ nicht erstellt.\\
	
	\item[\boldmath$max_{Z}$] Die maximale Zweigtiefe eines Knotens. Bevor ein neuer Knotenpunkt $v'$ als Nachfolger eines Knotenpunkts $v$ erstellt wird, wird seine Zweigtiefe $Z(v')$ berechnet. Übersteigt diese $max_{Z}$, wird der Knotenpunkt $v'$ nicht erstellt.\\
\end{description}

Weiterhin wurde gewichtetes Wachstum eingeführt -- eine Möglichkeit die Schrittweite $D$ in Abhängigkeit von der Zweigtiefe zu erhöhen. Befindet sich ein Knotenpunkt auf der Zweigtiefe $0$, werden neu hinzugefügte Nachfolger im Abstand von $2 * D$ positioniert, befindet er sich auf der maximalen Zweigtiefe, werden neu hinzugefügte Nachfolger im Abstand von $D$ positioniert. Zwischen diesen beiden Zweigtiefen wird die Schrittweite linear interpoliert. \\
 
Kurvenreduktion in Abhängigkeit des Abzweigungswinkel ermöglicht es die Baumstruktur mithilfe einer verringerten Datenmenge darzustellen. Besitzt ein Knoten $v$ genau einen Nachfolger $v_n$ und einen Vorgänger $v'$, werden die zwei Richtungsvektoren zwischen $v'$ und $v$ sowie $v$ und $v_n$ berechnet:
\begin{equation}
\overrightarrow{R_1} = \dfrac{p_v - p_{v'}}{\lVert p_v - p_{v'} \rVert} \text{ und } \overrightarrow{R_2} = \dfrac{p_{v_n} - p_{v}}{\lVert p_{v_n} - p_{v} \rVert}
\end{equation}

Übersteigt das Skalarprodukt $\langle \overrightarrow{R_1}, \overrightarrow{R_2} \rangle$ einen festgelegten Maximalwert, wird $v$ aus dem Baum entfernt. $v_n$ wird zum Nachfolger von $v'$ und $v'$ zum Vorgänger von $v_n$. 