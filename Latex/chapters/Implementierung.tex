\chapter{Implementierung}


Die Implementierung des Projekts lässt sich in vier Bereiche unterteilen: Die Baumrepräsentation enthält Daten, die von den L-System und Space Colonization Implementierungen generiert werden. Diese Daten werden an das Modellgenerierungssystem übergeben, welches die Modelldaten für eine grafische Darstellung in der Unreal Engine 4 produziert.


\section{Baumrepräsentation}

Sowohl L-Systeme als auch der Space Colonization Algorithmus generieren einen graphentheoretischen Baum, auf Grundlage dessen die Modellgenerierung durchgeführt wird. Die implementierte Baumrepräsentation kann daher von beiden Systemen verwendet werden und ermöglicht es, diese mit demselben Modellgenerierungssystem zu visualisieren.

Der Baum wird durch eine Datenklasse repräsentiert, jedes Objekt dieser Klasse beschreibt einen Knoten sowie die Kante, welche vom Vorgänger zu dem Knoten führt. Die Datenklasse bietet Zugriff auf die folgenden Informationen:

\begin{description}
	\item \textbf{Vorgänger und Nachfolger:} Mithilfe eines Verweises auf den Vorgänger und eine Liste der Nachfolger eines Knotens kann der Baum-Graph vollständig repräsentiert werden. Weiterhin ermöglicht dies die Implementierung einer Reihe von rekursiven Funktionen zur Anpassung von Modelldaten.\\
	
	\item \textbf{Modell-Daten:} Kanten werden, wie in Abschnitt \ref{subsec:ZylinderMeshes} beschrieben, mithilfe von Zylindern visualisiert. Um die Generierung von Modelldaten zu vereinfachen, bietet die Datenklasse Zugriff auf Start- und Endposition, Start- und Endradius, Start- und Endnormale sowie einen Rotationswinkel. 
	
	Weiterhin wird die Zweigtiefe des repräsentierten Knoten gespeichert.\\
	
	\item \textbf{Wachstums-Daten:} Die Wachstums-Daten bestehen aus einer Wachstumsrichtung, einem Einfluss-Zähler und dem \glqq Kein Wachstum\grqq-Zähler ($NG$-Counter), welche für den Ablauf des Space Colonization Algorithmus benötigt werden.	
\end{description}

Ein Objekt der Datenklasse kann als Astsegment eines biologischen Baumes angesehen werden und wird durch das Modellgenerierungssystem als solches visualisiert. Im Folgenden wird der Begriff \glqq Zweigobjekt\grqq{} verwendet, um ein Objekt der Baumrepräsentationsklasse zu bezeichnen.

\section{L-Systeme}

Die Implementierung einer L-System-Baumstruktur wird durch einen Unreal-Akteur verwirklicht, der im Level platziert werden kann. Nach Start des Levels wird das angegebene Axiom anhand der Produktionsregeln abgeleitet und die sich ergebende Zeichenkette von der Turtle-Implementierung interpretiert.

\subsection{Parameter}

Dem L-System-Akteur werden die folgenden Parameter über die Editor-UI übergeben:

\begin{description}
	\item \textbf{Anzahl der Ableitungen:} Die Anzahl der Ableitungen in $\mathbb{N}^+$, die auf dem Axiom durchgeführt werden sollen. \\
	
	\item \textbf{Axiom:} Das Axiom in Form einer Zeichenkette. \\
	
	\item \textbf{Konstanten:} Eine Konstante besteht aus der Angabe eines Identifikationssymbols und eines Wertes in $\mathbb{R}$. Konstanten können im Axiom und in den Nachfolgern der Produktionen verwendet werden.\\	
	
	\item \textbf{Produktionen:} Jede Produktion besteht aus Angabe eines Vorgängers, eine Liste von Parametern und einem Nachfolger. Der Vorgänger und jeder Parameter entspricht einem einzelnen Symbol, der Nachfolger wird als Zeichenkette eingetragen. Die Parametersymbole können nur innerhalb des Nachfolgers verwendet werden. Der Vorgänger und die Liste der Parameter bilden das parametrische Wort, welches bei einer Ableitung durch den Nachfolger ersetzt wird. \\
	
	\item \textbf{Tropismus:} Der Einfluss von Tropismus in Form eines dreidimensionalen Vektors und eines Biegsamkeitswertes.
\end{description}
\begin{figure} [hbtp]
	\centering
	\includegraphics[height=0.4\textheight]{images/LS_ExampleUE4UI.png}
	\caption{Ein Beispiel für die Angabe des L-Systems aus Gleichung \ref{eq:ProdBranching2} mit der resultierenden Baumstruktur aus Abbildung \ref{fig:Branching2L15D25}.}
	\label{fig:LS_ExampleUE4UI}
\end{figure}
Für das Axiom und die Produktionen gelten die in Kapitel \ref{ch:LSysteme} festgelegten Regeln für die Definition von L-Systemen. Weiterhin gelten die üblichen Regeln für die Angabe von arithmetischen Operationen. Die Verwendung von Klammern ist jedoch auf die Angabe von Parametern eines parametrischen Wortes beschränkt, ihre Verwendung zur Beeinflussung der Auswertungsreihenfolge eines arithmetischen Ausdrucks wird nicht unterstützt.

Die Parameter für eine Anpassung der Visualisierung der Turtle-Interpretationen werden in Abschnitt \ref{subsec:Modellgenerierung_Parameter} erläutert.

Ein Beispiel für die korrekte Eingabe eines L-Systems über die Editor-UI wird in Abbildung \ref{fig:LS_ExampleUE4UI} gezeigt.

\subsection{Ableitung}

Zu Anfang der Erstellung des L-System-Akteurs werden alle Konstantensymbole im Axiom und den Produktionen durch die Konstantenwerte ersetzt. 

Die Implementierung arbeitet durchgehend auf derselben Zeichenkette, angefangen mit dem Axiom. In jeder Ableitung werden die in den Produktionsregeln definierten parametrischen Wörter durch die angegebenen Nachfolger ersetzt.

Nachdem die vorgegebene Anzahl von Ableitungen durchgeführt wurde, wird die resultierende Zeichenkette an die Turtle-Implementierung weiter gegeben.

\subsection{Turtle Interpretation} \label{subsec:TurtleInterpretationImplementation}

Die Turtle-Implementierung bekommt die aus den Ableitungen resultierende Zeichenkette übergeben. Diese wird daraufhin sequentiell abgearbeitet und entsprechend den in Abschnitt \ref{sec:LS_Baumstrukturen} vorgestellten Konzepten interpretiert. Der resultierende Baum wird für die Konstruktion des Modells an das Modellgenerierungssystem weitergegeben.

\section{Space-Colonization Algorithmus}

Die Implementierung einer Space-Colonization-Baumstruktur stellt sich aus der Platzierung von mindestens einem Akteur für die Repräsentation des Einflussbereichs und einem Akteur für die Umsetzung des Algorithmus zusammen.

\subsection{Parameter}

Dem Space-Colonization-Akteur werden Parameter des ursprünglichen Algorithmus sowie die Eingaben für in Abschnitt \ref{sec:SCA_Erweiterungen} besprochene Erweiterungen über die Editor-UI übergeben. Zu den ursprünglichen Parametern gehören der Minimalradius, Einflussradius, Schrittweite, ein Tropismusvektor sowie die Anzahl der durchzuführenden Iterationen. Zu den erweiterten Parametern gehören der maximale Grad, die maximale Zweigtiefe, die maximale Anzahl von \glqq Kein Wachstum\grqq{}-Iterationen und eine Abfrage, ob gewichtetes Wachstum durchgeführt werden soll.

Weiterhin müssen die zugeordneten Einflussbereich-Akteure angegeben werden -- damit der Algorithmus durchgeführt werden kann, muss mindestens einer dieser Akteure mit mindestens einem Einflusspunkt eingetragen werden.

\subsection{Einflussbereiche}

Durch die Platzierung von Einflussbereich-Akteuren kann die Verteilung der Einflusspunkte mithilfe des Unreal-Editors beeinflusst werden. Es sind derzeit zwei Formen von Einflussbereichen wählbar: Eine Kugel- und eine Zylinderform. Die Kugelform benötigt die Eingabe eines Kugelradius während die Zylinderform durch eine Höhe und einen Radius beschrieben wird. 

Einem Space-Colonization-Akteur können mehrere Einflussbereiche zugeordnet werden, um eine bestimmte Baumstruktur zu formen. Jeder Einflussbereich-Akteur wird weiterhin in einem vorgegebenen Abstand von dem Space-Colonization-Akteur platziert.

Dem Einflussbereich muss eine positive Anzahl von zu generierenden Einflusspunkten und ein Random Seed Wert übergeben werden. Ein Random Seed ist ein Wert, der von einem Zufallsgenerator verwendet wird, um eine Folge von zufälligen Zahlen zu generieren. Bei Verwendung desselben Random Seed wird dieselbe Folge von Zufallszahlen erstellt, was eine Kontrolle über Generierung ermöglicht. Somit wird, wenn auch alle anderen Parameter übereinstimmen, mit demselben Random Seed Wert dieselbe Baumstruktur aufgebaut. 

\subsection{Ablauf des Algorithmus}

Die Einflusspunkte aller dem Space-Colonization-Akteur zugeordneten Einflussbereiche werden diesem zu Beginn der Baum-Generierung übergeben. Daraufhin wird ein Baum entsprechend der in Abschnitt \ref{sec:GenerierungBaumstrukturen} und Abschnitt \ref{sec:SCA_Erweiterungen} vorgestellten Konzepten aufgebaut und für die Konstruktion des Modells an das Modellgenerierungssystem weitergegeben.

\section{Modellgenerierung} \label{sec:Modellgenerierung}

Das Modellgenerierungssystem erhält einen graphentheoretischen Baum von L-System- und Space-Colonization-Akteuren und generiert ein dreidimensionales Mesh in der von der Unreal Engine geforderten Form. Das Mesh entspricht der Visualisierung der Kanten des Baums in Form von Zylindern und simuliert dadurch vereinfacht die Aststruktur eines biologischen Baumes. \cite[Abschn. 2]{SpaceColonizationAlgorithm:07} 

\subsection{Procedural Mesh Component}

Die Procedural Mesh Component ist eine Komponente der Unreal Engine, welche die Darstellung von prozedural generierten Polygonnetzen zulässt. Der Komponente werden Vertexdaten in Form von Listen aus Positions-, Normalen-, Tangenten-, Textur- und Indexdaten übergeben. Das Grafiksystem der Unreal Engine ist daraufhin in der Lage, die Komponente als dreidimensionales Modell im Level darzustellen. \cite{ProceduralMeshComponent:15} Die Vertexdaten werden, basierend auf dem übergebenen Baum und den Parametern, vom Modellgenerierungssytem erstellt.

Jedem L-System-Akteur und Space-Colonization-Akteur ist eine Procedural Mesh Component zugeordnet.

\subsection{Parameter} \label{subsec:Modellgenerierung_Parameter}

Dem Modellgenerierungssystem werden die folgenden Parameter über die Editor-UI übergeben:

\begin{description}
	\item \textbf{Radius-Daten:} Dies beinhaltet den Blattradius, den Radiuswachstumswert, den Stammbreitenmultiplikator und die Abfrage, ob Radiusberechnungen durchgeführt werden sollen. \\
	
	\item \textbf{Genauigkeit:} Dies beinhaltet die minimale und maximale Anzahl von Zylindersektionen sowie den Kurvenreduktionswert. \\
	
	\item \textbf{Sonstiges:} Weiterhin wird die Rotation jedes Zylindersegments auf der lokalen Z-Achse, ein Material und eine Abfrage, ob ein fraktales Mesh erstellt werden soll, übergeben. Ein Material beinhaltet Textur- und Shaderinformationen und ist für die Oberflächenbeschaffenheit des generierten Modells verantwortlich.
\end{description}

\subsection{Operationen auf dem Baum}

Folgende Operationen werden vor Beginn der Modelldatengenerierung auf dem graphentheoretischen Baum durchgeführt:

\begin{description}
	\item \textbf{Kurvenreduktion:} Die Kurvenreduktion wird, beginnend mit dem Wurzel-Zweigobjekt des Baums, rekursiv ausgeführt. In jedem Schritt wird überprüft ob, entsprechend der Beschreibung in Paragraph \ref{par:Kurvenreduktion}, das aktuelle Zweigobjekt entfernt werden kann. Falls nicht, wird die Kurvenreduktion auf allen Nachfolgern des aktuellen Zweigobjekts durchgeführt. Die Rekursion bricht ab, falls ein Zweigobjekt keine Nachfolger besitzt.
	
	Der Parameter \glqq Kurvenreduktionswert\grqq{} entspricht dem Maximalwert des Skalarprodukts $max_K$.\\
	
	\item \textbf{Radiusberechnung:} Der Endradius jedes Zweigobjekts wird, beginnend mit dem Wurzel-Zweigobjekt des Baums, rekursiv anhand von Gleichung \ref{eq:Radiusberechnung} berechnet. Der Parameter \glqq Blattradius\grqq{} entspricht $r_0$ und der \glqq Radiuswachstumswert\grqq{} entspricht $g$. 
	
	Der Startradius jedes Zweigobjekts wird auf den Wert des Endradius seines Vorgängers gesetzt. Da das Wurzel-Zweigobjekt keinen Vorgänger besitzt, wird der Startradius aus der Multiplikation des Stammbreitenmultiplikator mit dem Endradius des Objekts bestimmt.\\
	
	\item \textbf{Verringerung der Abzweigungswinkel:} Die Nachfolger jedes Zweigobjekts werden, entsprechend der Beschreibung in Paragraph \ref{par:VerringerungAbzweigungswinkel}, einander angenähert. Die Verringerung der Abzweigungswinkel wird nur bei einem durch einen Space-Colonization-Akteur generierten Baum durchgeführt.
\end{description}


\subsection{Generierung der Zylinder-Meshes} \label{subsec:ZylinderMeshes}

Grundlegend: Zylinder besteht aus zwei Ringen

Ring wird mithilfe eines Radius, einem Ringzentrum und einer Ringnormale beschrieben.

Die Ring-Modelldaten werden aus einem Rotationswinkel um die lokalen Z-Achse und einer vorgegebenen Anzahl von Segmenten erstellt. --> Positionen, Normalen, Tangenten und Texturdaten können berechnet werden.

Die beiden Ringe werden miteinander verbunden -> index daten korrekt berechnen

Erweiterung der Zylinder auf eine Beschreibung durch mehr als zwei Ringe, um weniger Vertexe generieren zu müssen --> Verwendung der Zweigtiefe um "Äste" zu erstellen


